\documentclass[final,hyperref={pdfpagelabels=false}]{beamer}

% Adjust the poster size
\usepackage[orientation=landscape, size=a0, scale=1.4]{beamerposter}

% Add additional packages as needed
\usepackage{lipsum}
\addtobeamertemplate{block begin}{\vspace*{10pt}}{}
\addtobeamertemplate{block end}{}{\vspace*{10pt}}
% Define the column layout
\usepackage[absolute,overlay]{textpos}
\setlength{\TPHorizModule}{1cm}
\setlength{\TPVertModule}{1cm}

% Title and authors
\title{Parallel Oblivious Sorting in Intel SGX Enclave}
\author{Tianyao Gu, Tian Xie}


\begin{document}
\begin{frame}
  \begin{columns}[t]

    % First column
    \begin{column}{.25\linewidth}
      \maketitle
      \begin{block}{Abstract}
        % Your abstract text here
        Oblivious algorithms have been a popular research topic in cybersecurity due to its resistance to side channel attacks. We implemented a parallel oblivious sorting algorithm in C++ using Intel SGX Enclave, a hardware-based trusted-computing environment. We leveraged OpenMP to achieve multithreading on a 36-core CPU and further improved parallelism through SIMD. We parallelized both the in-enclave computation and the communication with untrusted memory. Our implementation achieves a speedup of 3.5x over the original serial implementation with 8 threads and 4.2x with 32 threads. As a byproduct, we also obtained an efficient parallel oblivious random shuffling algorithm, which features and 14x speedup with 32 threads.
      \end{block}

      \begin{block}{Background of Intel SGX}
        % Background of Intel SGX text here
        Intel SGX is a hardware-based trusted-computing environment, which provides a protected region of memory, known as the Enclave Page Cache (EPC). EPC has limited size, and the communication between EPC and untrusted memory is time-consuming. Therefore, it is important to minimize the amount of data swapped in and out of EPC.
      \end{block}

      \begin{block}{Background of Oblivious Sorting Algorithms}
        % Background of Oblivious Sorting Algorithms text here
      Oblivious algorithms ensure that memory access and page swap patterns are independent of secret data, thereby resists side channel attacks.
      \end{block}

      \begin{block}{Background of Flex-way Butterfly Oblivious Sort}
        % Background of Flex-way Butterfly Oblivious Sort text here
        To achieve oblivious sorting, Flex-way Butterfly Oblivious Sort applies a random shuffling algorithm followed by a non-oblivious comparison-based sorting algorithm. The key component of the shuffling algorithm is a multi-way butterfly network, emulated with a building block called Merge-split. The algorithm ensures negligible overflow probability and achieves optimal complexity in terms of both computation and I/O.
        \begin{figure}
          \includegraphics[width=\linewidth]{assets/multi-way-butterfly.png}
          \caption{A Multi-way Butterfly Network}
        \end{figure}
      \end{block}

    \end{column}

    % Third column
  \begin{column}{.25\linewidth}
  \begin{block}{Parallel Oblivious Sorting Implementation Overview}
    \end{block}
    \begin{figure}
      \includegraphics[width=\linewidth]{assets/parosort.png}
      \caption{Diagram of Parallel Oblivious Sort on a toy example. Each color represents a thread. Colorful components are multi-threaded.}
    \end{figure}
  \end{column}

    % Second column

  \begin{column}{.30\linewidth}

    \begin{block}{Speedup Diagrams for separtate optimizations}
    \end{block}
    \begin{figure}[h]
        \centering
        % \begin{minipage}{0.48\linewidth}
        %     \includegraphics[width=\linewidth]{assets/speedup_shuffle.png}
        %     \caption{Speedup for Oblivious Shuffling}
        % \end{minipage}
        % \hfill
        \begin{minipage}{0.8\linewidth}
            \includegraphics[width=\linewidth]{assets/shuffle.png}
            \caption{Runtime Improvement Through SIMD}
        \end{minipage}
    \end{figure}

    \begin{figure}[h]
        \centering
        \begin{minipage}{0.48\linewidth}
            \includegraphics[width=\linewidth]{assets/sorting.png}
            \caption{Speedup for External MergeSort with Multi-threaded I/O}
        \end{minipage}
        \hfill
        \begin{minipage}{0.48\linewidth}
            \includegraphics[width=\linewidth]{assets/sort.png}
            \caption{Speedup using parallel sorting compared to std::sort}
        \end{minipage} \\ \\  \\
    \end{figure}
    \vspace{50}
    \begin{block}{Overall Speedup Diagram for Our Parallel K-Way Butterfly Networking Oblivious Sorting/Shuffling}
    \end{block}
    % Overall Speedup Diagram for our Parallel K-Way Butterfly Networking Oblivious Sorting \\
    % \begin{figure}
    %   \includegraphics[width=0.8\linewidth]{assets/speedup_sort.png}
    %   \caption{The Overall Speedup of Oblivious Sorting Algorithm}
    % \end{figure}
    \begin{figure}[h]
        \centering
        \begin{minipage}{0.48\linewidth}
            \includegraphics[width=\linewidth]{assets/speedup_shuffle.png}
            \caption{Speedup for Oblivious Shuffling}
        \end{minipage}
        \hfill
        \begin{minipage}{0.48\linewidth}
            \includegraphics[width=\linewidth]{assets/speedup_sort.png}
            \caption{Speedup for Oblivious Sorting}
        \end{minipage}
    \end{figure}

    % \begin{figure}
    %   \includegraphics[width=0.6\linewidth]{assets/sorting.png}
    %   \caption{Diagram of Parallel Oblivious Sort on a toy example. Each color represents a thread. Colorful components are multi-threaded.}
    % \end{figure}
  \end{column}
  
  \end{columns}
\end{frame}
\end{document}
